В отличии от модели компонентных объектов C\-O\-M данная архитектура не подразумевает глобальное использование объектов в операционной системе. А предоставляет основу для инструментов позволяющих конструировать объекты из имеющихся компонентов и дополнять их функционал. Основой данной архитектуры является базовый объект среды в которой она применяется. Базовый объект среды это контейнер компонентов, который агрегирует в себе их свойства и обеспечивает коммутацию между ними, являесь их менеджером.

Каждый контейнер обеспечивает доступ к своим компонентам благодаря уникальной ассоциации между классом компонента и объектом данного класса. Таким образом каждый контейнер может содержать только один объект своего класса.

Для соотнашения нескольких объектов одному классу для контейнера необходимо использовать функции для работы со смешанным ассоциативным списком. Это приводит к дополнительным расходам и не подходит для быстрого получения доступа к компонентам. 